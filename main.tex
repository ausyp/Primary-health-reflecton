
\documentclass[a4paper,man,british]{apa6}
\usepackage[british]{babel}
\usepackage[utf8]{inputenc}
\usepackage{csquotes}
\usepackage[hidelinks]{hyperref}
%\usepackage[style=apa]{biblatex}
%\DeclareLanguageMapping{british}{british-apa}

% maps apacite commands to biblatex commands
%\let \citeNP \cite
%\let \citeA \textcite
%\let \cite \parencite

% disclosure footnote




\title{Reflective Essay : A Reflection of group project experience and impression}
\shorttitle{Reflective Essay on group Project}



\affiliation{RMIT UNIVERSITY}




\begin{document}

\maketitle

\section{Introduction}
 “\textit{life is about the journey and not the destination}”  is an unattributed quote that owns its relevance to its self-evidence and in that spirit, we hope to elaborate on our journey as a group in the projects we undertook as part of this course. Through this essay, we aim to discuss the pinnacles of our experience and the unavoidable low points we found ourselves in. This reflection is also an attempt to examine what we would have done differently had we know what we know now.

\section{Our experience as a group}

Properly structured group work is essential to get tasks done. The outstanding fact about this group project was the equal participation and involvement of each individual in bringing about the best outcome. After a discussion between the team members, in order to decrease the workload, the project was divided equally among everyone. All the team members came together and combined their work into one and helped others who were struggling with their part. Due to equal effort, we were able to publish the finest result. We were able to bring up all our talents based on our strengths and helped others by considering their weaknesses. Some of our team members were well organized when compared with others which helped to lead the path, to organize the project and to manage the time accordingly. Group decision-making in our team often requires compromise, and sometimes a willingness to give up one’s point of view in favor of the group's shared views and decisions.
The project provided us the ability to work as a team and gave the chance to interact with other members and acquire more information about Drug Abuse. Drug abuse being one of the main issues in society has to be considered and brought up into the notice of the representatives. We were able to concentrate on the issue in the Yarra City council by referring the journals, articles, and books which helped us to achieve more data and information about the case. 

\section{Strengths of working as a group}
Teamwork is crucial to facilitate effective nursing communication and to provide well-organized patient care. Teamwork is an inevitable part of nursing profession as nurses work closely alongside clinicians and specialists to provide patient care. Communication strategies help nurses to build a strong foundation for successful healthcare outcomes. With the help of the team members, we were able to create a plan and time frame to achieve the best outcome. A couple of our members lead the group by organizing the work and managing time which helped us to pull out a better end result. We trusted each other blindfolded and everyone was able to complete the project much better than we expected. Each of us came to observe what we are capable of and to exceed our capacity and come out of our comfort zone. We had the aim of completing the project within a couple of weeks before the due date and coming together to combine our work so that we can make the changes that have to be made and help others if they were struggling to complete their work. 
Due to some unavoidable circumstances, our team members were not able to meet physically to discuss the progression of our individual work. So, we used google docs, slack and social media to share individual work with each team member and had a chance to look at each other's work. We were able to edit and add more points on the project after discussing it with the team through slack. 

\section{Challenges of working on a group project}

The Challenges we faced in this project were akin to challenges faced by teams in the real world. As in the real-world,  our overall speed was determined by the slowest individual factor as it determines the rate at which a group can operate. A hurdle we faced was friction between team members as individual members' works stand on the shoulders of their colleague's work and if one member delays the delivery of their work, the whole project suffers set back.
 Due to physical limitations, our project was co-ordinated using group video conferences and slack channels even though this strategy was individually effective and successful it wasn't as good as having a team in close proximity working together we were  only able to work together in the same room in the last days and we discovered things we could have improved upon.
Another hickup we faced was integrating multiple people's works into a clear concise essay with a uniform standard. Another challenge we faced was the varying spectrum engagement from different team members. Some team members who go the extra mile may feel burned out due to a lack of engagement from their collaborators.

\section{What we would do differently}

Had we known the things we understand now, we would have chosen to do somethings differently:

We would have placed more importance to face to face meeting and would have tried to work as a group in close proximity where we may have been able to foster an environment where it is impossible to stay detached and achieve a uniform engagement from all team members. We would in the future, make sure to have regular check-ins with each other and peer review our works, as the model of deligating individual workloads with a uniform deadline, we adopted has its drawbacks.
%\newpage
\section{Conclusion}
We had our strength in our ability to organize and plan the structure and topic of our projects.  Our workloads were fairly distributed among participants
Even though we had our difficulties and things that we may have done differently, in the end, we were able to bring out the best in every one of us.

\end{document}
